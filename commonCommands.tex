\usepackage [utf8] {inputenc} 
\usepackage [italian] {babel}

\usepackage {xcolor}
\usepackage {geometry}

\usepackage {amsmath}
\usepackage {amssymb}
\usepackage {amsfonts}

\usepackage {mathrsfs}
\usepackage {mathtools}
\usepackage {faktor}

\usepackage {stmaryrd}
\usepackage [colorlinks=true, allcolors=black] {hyperref}


\geometry{hmargin={1.5cm,1cm}, vmargin={1.5cm, 1.5cm}}

%Tabbing structure
\newcommand {\ftab} {\hspace*{1cm}}
\newcommand {\ntab} {\hspace{1cm}}
\newcommand {\mtab} {\hspace{0.5cm}}
\newcommand {\dtab} {\hspace{2cm}}
\newcommand {\ptab} {\ftab\=\mtab\=\mtab\=\ntab\=\ntab\=\ntab\=\ntab\=\ntab\=\ntab\=\ntab\=\ntab\=\dtab\=\dtab\=\\}
\newcommand {\stab} {\begin{tabbing}}
\newcommand {\etab} {\end{tabbing}}
\renewcommand {\t}[0] {\>}
\newcommand {\n} [0] {\\}

\newcommand {\tdx}  [0] {\t \t}

%Custom labels
\newcommand {\defg} 	[0] {\textbf{Definizione}: }
\newcommand {\notg} 	[0] {\underline{Notatione}: }
\newcommand {\teog} 	[0] {\textbf{Teorema}: }
\newcommand {\lem} 	[0] {\textbf{Lemma}: }
\newcommand {\cor} 		[0] {\tdx\textbf{Corollario}: }
\newcommand {\prop} 	[0] {\underline{Proprietà}: }
\newcommand {\dimg} 	[0] {\textit{Dimostrazione}: }


%Definitions -> move to a separate file
\newcommand{\stackHookR}{%
  \mathrel{%
    \vcenter{%
      \offinterlineskip
      \hbox{\makebox[1.5em][c]{$\scriptstyle\hookrightarrow$}}%
      \kern-1.0ex
      \hbox{\raisebox{0.2ex}{$\xhookrightarrow{\hspace{1em}}$}}%
    }%
  }%
}

%Capitol mathematical letters
\newcommand {\N}	[0] {\mathbb{N}}	%Natural numbers
\newcommand {\Z} 	[0] {\mathbb{Z}}		%Integers
\newcommand {\Q} 	[0] {\mathbb{Q}} 	%Rational numbers
\newcommand {\R} 	[0] {\mathbb{R}}   	%Real numbers
\newcommand {\C} 	[0] {\mathbb{C}}   	%Complex numbers
\renewcommand {\H}	[0] {\mathbb{H}} 	%Quaternions
\newcommand {\K} 	[0] {\mathbb{K}}		%Generic field
\newcommand {\F} 	[0] {\mathbb{F}}		%Generic finite field

\newcommand {\Tsp}	[0] {\mathcal{D}}	%Spazio delle funzioni di test
\newcommand {\dist}	[0] {\Tsp'}			%Spazio delle distribuzioni
\newcommand {\cnt}	[0] {\mathcal{Z}}	%Center of an algebraic structure

\newcommand {\prt}	[0] {\mathscr{P}}	%Power set 

\renewcommand {\S} [0] {\mathbb{S}}		%Sphere


%Category names
\newcommand {\gp}	[0] {Group}
\newcommand {\rg}	[0] {Ring}
\newcommand {\id}	[0] {Ideal}
\newcommand {\st}	[0] {Set}
\newcommand {\fd}	[0] {Field}
\newcommand {\ag}	[0] {Alg}
\newcommand {\vt}	[0] {Vect}
\newcommand {\rp}	[0] {Rep}
\newcommand {\irr}	[0] {Irr}
\newcommand {\idc}	[0] {Indec}

%Arrows
\newcommand {\imp} 	[0] {\implies}							%Implication
\newcommand {\ind} 		[0] {\stackrel{ind}{\imp}}					%Implication by induction

\newcommand {\conv} 	[0] {\xrightarrow{\phantom{....}}}			%Basic convergence
\newcommand {\convIn} 	[1] {\xrightarrow{#1}}					%Labeled convergence
\newcommand {\convU} 	[0] {\convIn{\text{unif.}}}					%Uniform convergence
\newcommand {\conW} 	[0] {\xrightharpoonup{\phantom{....}}}		%Basic weak convergence
\newcommand {\conWIn}	[1] {\xrightharpoonup{#1}}				%Labeled weak convergence
\newcommand {\conS}	[0] {\conWIn{\phantom{.}*\phantom{.}}}	%Basic weak* convergence
\newcommand {\conSIn}	[1] {\conWIn{#1 - *}}					%Labeled weak* convergence

\newcommand {\into}		[0] {\xhookrightarrow{\phantom{....}}}		%Basic simple immersion
\newcommand {\intc}		[0] {\stackHookR}						%Basic compact immersion

\newcommand {\morseq}	[1] {\stackrel{#1}{\longrightarrow}}		%...

%Math custom operators
\DeclareMathOperator*{\Sup} {sup}		%Supremun				
\DeclareMathOperator*{\Inf} {inf}			%Infimun
\DeclareMathOperator{\sgn}{sgn}			%Sign function

\DeclareMathOperator{\im} {im}			%Image of a function
\DeclareMathOperator{\supp} {supp}		%Support of a function

\DeclareMathOperator*{\Id} {Id}			%Identity function
\DeclareMathOperator*{\sech} {sech}		%Hyperbolic secant
\DeclareMathOperator*{\csch} {csch}		%Hyperbolic cosecant

\DeclareMathOperator{\opdiv} {div}		%Divergence
\renewcommand {\div} [0] {\opdiv}

\DeclareMathOperator {\mres} {\lfloor}		%Measure restriction
\DeclareMathOperator {\D} {D}			%Distributional derivative

\newcommand {\Dac} {\D^{\text{ac}}}		%BV derivative - absolute continum part
\renewcommand {\DJ} {\D^{\text{J}}}		%BV derivative - jump part
\newcommand {\DC} {\D^{\text{C}}}		%BV derivative - Cantor part

\DeclareMathOperator*{\curl} {curl}			%curl of a vector field
\DeclareMathOperator*{\lowerJ} {j}			%???
\renewcommand {\j} [0] {\lowerJ}			
\DeclareMathOperator {\J} {J}				%Jacobian

%Math inline symbols
\renewcommand {\(}		[0] {\left(}							%Adaptive left round bracket
\renewcommand {\)}		[0] {\right)}							%Adaptive right round bracket
\renewcommand {\[}		[0] {\left\llbracket}					%Adaptive left square bracket
\renewcommand {\]}		[0] {\right\rrbracket}					%Adaptive right square bracket
\newcommand {\la}		[0] {\left\langle}						%Adaptive left angle bracket
\newcommand {\ra}		[0] {\right\rangle}					%Adaptive right angle bracket

\newcommand {\dif} 		[0] {\setminus}						%Set difference
\newcommand {\difsim} 	[0] {\triangle}						%Set simmetric difference

\newcommand {\dirsum} 	[0] {\oplus}							%Direct sum
\newcommand {\somort} 	[0] {\stackrel{\bot}{\dirsum}}			%Ortogonal direct sum
\newcommand {\tnsr} 	[0] {\otimes}							%Tensor product

\newcommand {\isom} 	[0] {\simeq}							%Isomorfismo
\newcommand {\ore} 	[0] {\stackrel{.}{\simeq}}				%...

\newcommand {\norm} 	[0] {\unlhd}							%Normal subgroup
\newcommand {\pint} 	[1] {\mathring{#1}}					%Internal part of a set

\newcommand {\derp} 	[0] {\partial}							%Partial derivative operator

\newcommand {\paral}	[0] {\;\mathbin{\!/\mkern-5mu/\!}\;}	%Parallel symbol

\newcommand {\tc}		[0] {\; : \;}							%s.t. symbol 
\newcommand {\bmid}	[0] {\;\Big|\;}						%big s.t. symbol for set definitions

\newcommand {\checkv}	[0] {\checkmark}						%v-chech symbol
\newcommand {\1}		[0] {1}								%Characteristic function

%Math displaystyle symbols
\newcommand {\som} 	[0] {\displaystyle\sum}				%Summation
\newcommand {\serl} 	[0] {\som_{n=-\infty}^{+\infty}}		%Series summation on Z
\newcommand {\union} 	[0] {\displaystyle \bigcup}				%Union
\newcommand {\inters} 	[0] {\displaystyle \bigcap}				%Intersection
\newcommand {\undis}	[0] {\displaystyle \bigsqcup}			%Disjoint union
\newcommand {\somdir}	[0] {\displaystyle \bigoplus}			%Direct sum

\newcommand {\mx} 		[0] {\displaystyle \max}				%Maximum
\newcommand {\mn} 	[0] {\displaystyle \min}				%Minimum
\renewcommand {\sup} 	[0] {\displaystyle\Sup}				%Supremum
\renewcommand {\inf} 	[0] {\displaystyle\Inf}					%Infimum

\newcommand {\limf} 	[0] {\displaystyle\lim}					%Generic limit
\newcommand {\lims} 	[0] {\displaystyle\lim_{n\to+\infty}}	%Sequence limit
\newcommand {\limSup}	[0] {\displaystyle \limsup}				%Generic limsup
\newcommand {\lms} 	[0] {\limSup_{n\to+\infty}}			%Sequence limsup
\newcommand {\limInf}	[0] {\displaystyle \limInf}				%Generic liminf
\newcommand {\lmi} 		[0] {\limInf_{n\to+\infty}}			%Sequence liminf

\newcommand {\fraz} 	[0] {\displaystyle\frac}				%Fractions
\newcommand {\intg} 	[0] {\displaystyle\int}					%Single integral
\newcommand {\iintg} 	[0] {\displaystyle\iint}					%Double integral
\newcommand {\iiintg} 	[0] {\displaystyle\iiint}				%Triple integral

\newcommand {\comp} 	[0] {\displaystyle\complement}			%Set complementary symbol


%Text modifiers

\newcommand {\ud}		[0] {\underline}
\newcommand {\ov}		[0] {\overline}