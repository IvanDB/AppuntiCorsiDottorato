\documentclass [a4paper] {article}

\usepackage [utf8] {inputenc} 
\usepackage [italian] {babel}

\usepackage {geometry}
\usepackage {xcolor}
\usepackage {xparse}
\usepackage {xstring}

\usepackage {amsmath}
\usepackage {amssymb}
\usepackage {amsfonts}
\usepackage {amsthm}
\usepackage {eqparbox}

\usepackage {mathrsfs}
\usepackage {mathtools}
\usepackage {faktor}
\usepackage {esint}

\usepackage {stmaryrd}
\usepackage [colorlinks=true, allcolors=black] {hyperref}
\usepackage {nameref}


\geometry{hmargin={1.5cm,1cm}, vmargin={1.5cm, 1.5cm}}

\newcommand {\n} [0] {\\}


\newenvironment{body} 
	{\par \begingroup \setlength{\leftskip}{4em} \setlength{\parindent}{0pt} \everypar={\setlength{\leftskip}{4em}\setlength{\parindent}{0pt}}}
	{\par \endgroup}

\newtheoremstyle{boldBlock}	{20pt}	{0pt} {\normalfont} {} {} {:} {.5em} {\textbf{\thmname{#1}}\thmnote{ [#3]}}
\newtheoremstyle{seqBlock}	{0pt}	{0pt} {\normalfont} {} {} {:} {.5em} {\textbf{\thmname{#1}}\thmnote{ [#3]}}
\newtheoremstyle{undBlock}	{0pt}  	{0pt} {\normalfont} {} {} {:} {.5em} {\underline{\thmname{#1}}\thmnote{ [#3]}}

\theoremstyle{boldBlock}
\newtheorem*	{customDef}	{Definizione}
\newtheorem*	{customTeo}	{Teorema}
\newtheorem*	{customLem}	{Lemma}
\newtheorem*	{customPrp}	{Proposizione}
\theoremstyle{seqBlock}
\newtheorem*	{customCor}	{Corollario}
\theoremstyle{undBlock}
\newtheorem*	{customProp}	{Proprietà}
\newtheorem*	{customNot}	{Notazione}

\newcommand {\argCheck} [3] {\if\relax\detokenize{#1}\relax #3 \else #2 \fi}

\NewDocumentCommand {\defc}	{O{} o m m}	{\begin{customDef} 	[#1] \IfValueT{#2}{\phantomsection\label{teo:#2}}	\argCheck{#3}{#3 \begin{body} #4 \end{body}}{#4}	\end{customDef}}
\NewDocumentCommand {\teoc}	{O{} o m m}	{\begin{customTeo}	[#1] \IfValueT{#2}{\phantomsection\label{teo:#2}}	\argCheck{#3}{#3 \begin{body} #4 \end{body}}{#4}	\end{customTeo}}
\NewDocumentCommand {\lemc}	{O{} o m m}	{\begin{customLem}	[#1] \IfValueT{#2}{\phantomsection\label{lem:#2}}	\argCheck{#3}{#3 \begin{body} #4 \end{body}}{#4}	\end{customLem}}
\NewDocumentCommand {\corc}	{O{} o m m}	{\begin{customCor}	[#1] \IfValueT{#2}{\phantomsection\label{cor:#2}} 	\argCheck{#3}{#3 \begin{body} #4 \end{body}}{#4}	\end{customCor}}
\NewDocumentCommand {\prop}	{O{} o m}	{\begin{customProp}	[#1] \IfValueT{#2}{\phantomsection\label{teo:#2}}	#3												\end{customProp}}
\NewDocumentCommand {\prpc}	{O{} o m m}	{\begin{customPrp}	[#1] \IfValueT{#2}{\phantomsection\label{lem:#2}}	\argCheck{#3}{#3 \begin{body} #4 \end{body}}{#4}	\end{customPrp}}
\NewDocumentCommand {\notz}	{O{} o m m}	{\begin{customNot}	[#1] \IfValueT{#2}{\phantomsection\label{teo:#2}}	\argCheck{#3}{#3 \begin{body} #4 \end{body}}{#4}	\end{customNot}}

\newcommand {\edim}	{\renewcommand{\qedsymbol}{}\begin{proof} ... \end{proof}}
\NewDocumentCommand {\dimc} {m} {\renewcommand{\qedsymbol}{$\square$}\begin{proof} \phantom{.} \begin{body} #1 \end{body}\end{proof}}
\newcommand {\tdim}	{\dimc{... to do ...}}

\newcommand {\defnm} [1] {\hfill\eqparbox{dx}{#1 \phantom{....................}}}
\newcommand {\defif} [1] {\defnm{se \hspace{1em} #1}}


%Definitions -> move to a separate file
\newcommand{\stackHookR}
	{\mathrel{\vcenter{\offinterlineskip \hbox{\makebox[1.5em][c]{$\scriptstyle\hookrightarrow$}} \kern-1.0ex \hbox{\raisebox{0.2ex}{$\xhookrightarrow{\hspace{1em}}$}}}}}

%Capitol mathematical letters
\newcommand {\N}	[0] {\mathbb{N}}	%Natural numbers
\newcommand {\Z} 	[0] {\mathbb{Z}}	%Integers
\newcommand {\Q} 	[0] {\mathbb{Q}} 	%Rational numbers
\newcommand {\R} 	[0] {\mathbb{R}}  %Real numbers
\newcommand {\C} 	[0] {\mathbb{C}}  %Complex numbers
\renewcommand {\H}	[0] {\mathbb{H}} 	%Quaternions
\newcommand {\K} 	[0] {\mathbb{K}}	%Generic field
\newcommand {\F} 	[0] {\mathbb{F}}	%Generic finite field

\newcommand {\Tsp}	[0] {\mathcal{D}}	%Test functions
\newcommand {\dist}	[0] {\Tsp'}		%Distributions
\newcommand {\M}	[0] {\mathcal{M}}	%Measure


\newcommand {\cnt}	[0] {\mathcal{Z}}	%Center of an algebraic structure

\newcommand {\prt}	[0] {\mathscr{P}}	%Power set 

\renewcommand {\S} [0] {\mathbb{S}}	%Sphere


%Category names
\newcommand {\gp}	[0] {Group}
\newcommand {\rg}	[0] {Ring}
\newcommand {\id}	[0] {Ideal}
\newcommand {\st}	[0] {Set}
\newcommand {\fd}	[0] {Field}
\newcommand {\ag}	[0] {Alg}
\newcommand {\vt}	[0] {Vect}
\newcommand {\rp}	[0] {Rep}
\newcommand {\irr}	[0] {Irr}
\newcommand {\idc}	[0] {Indec}

%Arrows and sign
\newcommand {\eqIn}		[1] {\stackrel{#1}{=}}							%Equals in	
\newcommand {\imp} 		[0] {\implies}									%Simple Implication
\newcommand {\ind} 		[0] {\stackrel{\text{ind}}{\imp}}					%Implication by induction
\newcommand {\impby} 	[1] {\stackrel{#1}{\imp}}						%Implication by something
\NewDocumentCommand {\impref} {o m} {\stackrel{\hyperref[\IfValueT{#1}{#1:}#2]{\text{#2}}}{\imp}}	%Implication by label

\newcommand {\conv} 		[0] {\xrightarrow{\phantom{....}}}				%Basic convergence
\newcommand {\convIn} 	[1] {\xrightarrow{#1}}							%Labeled convergence
\newcommand {\convU} 	[0] {\convIn{\text{unif.}}}						%Uniform convergence
\newcommand {\conW} 	[0] {\xrightharpoonup{\phantom{....}}}			%Basic weak convergence
\newcommand {\conWIn}	[1] {\xrightharpoonup{#1}}						%Labeled weak convergence
\newcommand {\conS}		[0] {\conWIn{\phantom{.}*\phantom{.}}}			%Basic weak* convergence
\newcommand {\conSIn}	[1] {\conWIn{#1 - *}}							%Labeled weak* convergence

\newcommand {\into}		[0] {\xhookrightarrow{\phantom{....}}}			%Basic simple immersion
\newcommand {\intc}		[0] {\stackHookR}								%Basic compact immersion

\newcommand {\morseq}	[1] {\stackrel{#1}{\longrightarrow}}				%...

%Math custom operators
\DeclareMathOperator*{\Sup} {sup}		%Supremun				
\DeclareMathOperator*{\Inf} {inf}			%Infimun
\DeclareMathOperator{\sgn}{sgn}			%Sign function

\DeclareMathOperator{\im} {im}			%Image of a function
\DeclareMathOperator{\supp} {supp}		%Support of a function

\DeclareMathOperator*{\Id} {Id}			%Identity function
\DeclareMathOperator*{\sech} {sech}		%Hyperbolic secant
\DeclareMathOperator*{\csch} {csch}		%Hyperbolic cosecant

\DeclareMathOperator{\opdiv} {div}		%Divergence
\renewcommand {\div} [0] {\opdiv}

\DeclareMathOperator {\mres} {\lfloor}		%Measure restriction
\DeclareMathOperator {\D} {D}			%Distributional derivative

\newcommand {\Dac} {\D^{\text{ac}}}		%BV derivative - absolute continum part
\renewcommand {\DJ} {\D^{\text{J}}}		%BV derivative - jump part
\newcommand {\DC} {\D^{\text{C}}}		%BV derivative - Cantor part

\DeclareMathOperator*{\curl} {curl}			%curl of a vector field
\DeclareMathOperator*{\lowerJ} {j}			%???
\renewcommand {\j} [0] {\lowerJ}			
\DeclareMathOperator {\J} {J}				%Jacobian

%Math inline symbols
\renewcommand {\(}		[0] {\left(}							%Adaptive left round bracket
\renewcommand {\)}		[0] {\right)}							%Adaptive right round bracket
\renewcommand {\[}		[0] {\left\llbracket}					%Adaptive left square bracket
\renewcommand {\]}		[0] {\right\rrbracket}					%Adaptive right square bracket
\newcommand {\la}		[0] {\left\langle}						%Adaptive left angle bracket
\newcommand {\ra}		[0] {\right\rangle}					%Adaptive right angle bracket

\newcommand {\dif} 		[0] {\setminus}						%Set difference
\newcommand {\difsim} 	[0] {\triangle}						%Set simmetric difference

\newcommand {\dirsum} 	[0] {\oplus}							%Direct sum
\newcommand {\somort} 	[0] {\stackrel{\bot}{\dirsum}}			%Ortogonal direct sum
\newcommand {\tnsr} 	[0] {\otimes}							%Tensor product

\newcommand {\isom} 	[0] {\simeq}							%Isomorfismo
\newcommand {\ore} 	[0] {\stackrel{.}{\simeq}}				%...

\newcommand {\norm} 	[0] {\unlhd}							%Normal subgroup
\newcommand {\pint} 	[1] {\mathring{#1}}					%Internal part of a set

\newcommand {\derp} 	[0] {\partial}							%Partial derivative operator

\newcommand {\paral}	[0] {\;\mathbin{\!/\mkern-5mu/\!}\;}	%Parallel symbol

\newcommand {\tc}		[0] {\; : \;}							%s.t. symbol 
\newcommand {\bmid}	[0] {\;\Big|\;}						%big s.t. symbol for set definitions

\newcommand {\checkv}	[0] {\checkmark}						%v-chech symbol
\newcommand {\1}		[0] {1}								%Characteristic function

%Math displaystyle symbols
\newcommand {\som} 	[0] {\displaystyle\sum}				%Summation
\newcommand {\serl} 	[0] {\som_{n=-\infty}^{+\infty}}		%Series summation on Z
\newcommand {\union} 	[0] {\displaystyle \bigcup}		%Union
\newcommand {\inters} 	[0] {\displaystyle \bigcap}		%Intersection
\newcommand {\undis}	[0] {\displaystyle \bigsqcup}			%Disjoint union
\newcommand {\somdir}	[0] {\displaystyle \bigoplus}		%Direct sum

\newcommand {\mx} 		[0] {\displaystyle \max}			%Maximum
\newcommand {\mn} 	[0] {\displaystyle \min}				%Minimum
\renewcommand {\sup} 	[0] {\displaystyle\Sup}			%Supremum
\renewcommand {\inf} 	[0] {\displaystyle\Inf}				%Infimum

\newcommand {\limf} 	[0] {\displaystyle\lim}				%Generic limit
\newcommand {\lims} 	[0] {\displaystyle\lim_{n\to+\infty}}	%Sequence limit
\newcommand {\limSup}	[0] {\displaystyle \limsup}		%Generic limsup
\newcommand {\lms} 	[0] {\limSup_{n\to+\infty}}			%Sequence limsup
\newcommand {\limInf}	[0] {\displaystyle \liminf}			%Generic liminf
\newcommand {\lmi} 		[0] {\limInf_{n\to+\infty}}		%Sequence liminf

\newcommand {\fraz} 	[0] {\displaystyle\frac}				%Fractions
\newcommand {\intg} 	[0] {\displaystyle\int}				%Single integral
\newcommand {\iintg} 	[0] {\displaystyle\iint}				%Double integral
\newcommand {\iiintg} 	[0] {\displaystyle\iiint}				%Triple integral
\newcommand {\fintg} 	[0] {\displaystyle\fint}				%Triple integral

\newcommand {\comp} [0] {\displaystyle\complement}		%Set complementary symbol

\newcommand {\floor}	[1] {\left\lfloor {#1} \right\rfloor}


%Text modifiers

\newcommand {\ud}		[0] {\underline}
\newcommand {\ov}		[0] {\overline}

\newcommand{\spd} {s.p.d. }
\newcommand{\wrt} {w.r.t. }
\newcommand{\st} {s.t. }

\title{\textbf{Krylov Subspace Methods and Preconditioninggit }}
\author{Ivan De Biasi}
\date{}

\begin{document}

\maketitle

\pagestyle{plain}
\tableofcontents
\newpage

\section{Classical methods}
\subsection{Stationary methods}
Given a non singular square matrix $A \in \R^{n\times n}$, a vector $b \in \R^n$ and a \textit{splitting} $A = B - C$ of $A$ the linear system 
$$Ax = b$$ 
is equivalent to the fixed-point problem
$$x = B^{1}Cx + B^{-1}b.$$
So, given an intial guess $x_0 \in \R^n$, the iterative method associated is given by the recursion
$$x_{k+1} =  B^{1}Cx + B^{-1}b = T x + c$$
where $T = B^{1}C$ and $c = B^{1}b$.
It holds the following
\prpc{$\|e_k\|_2 \convIn{k\to \infty} 0 \iff \rho(T) <1$}{}

\subsection{Steepest descend (Gradient descend)}
Given a linear system $Ax = b$, if $A$ is \spd the solution $x^*$ satisfies
$$x^* = \argmin_{x \in \R^n} J(x) \qquad \text{where} \qquad J(x) = \frac{1}{2} x^\trasp A x - b^\trasp x.$$
The recursion given by
$$x_{k+1} = x_k + \alpha_k r_k \qquad \text{where} \qquad \alpha_k = \argmin_{\alpha \in \R} J(x_k + \alpha r_k) = \dots = \fraz{r_k^\trasp r_k}{r_k^\trasp Ar_k}$$
satisfies the following 
\teoc{$\forall k \quad \|e_k\|_A \leq \(\fraz{\kappa_2(A) - 1}{\kappa_2(A) + 1}\)^k\|e_0\|_A$}{}


\section{Krylov methods}
\subsection{Krylov space}
\defc{Given a square matrix $A \in \K^{n\times n}$ and a vector $v \in \K^n$ the \textit{$m^{th}$ (polynomial) Krylov subspace} of $A$ and $v$ is}
	{$$\mathcal{K}_m(A, v) = \vspan\{v, Av, \dots A^{m-1}v\}$$}
	
\defc{Given a square matrix $A \in \K^{n\times n}$ and a vector $v \in \K^n$ the \textit{minimal polynomial of $v$ \wrt $A$} is the monic polynomial $p$ with the lowest degree \st $p(A) = 0$. The degree $\deg_A(v)$ of such $p$ is called the \textit{grade of $v$ \wrt $A$}.}{}

\phantom{.}\n\n
The following statements holds
\prpc{Given a square matrix $A \in \K^{n\times n}$, a vector $v \in \K^n$ and $\mu = \deg_A(v)$ the grade of $v$ \wrt $A$ the $\mu^{th}$ Krylov subspace $\mathcal{K}_\mu(A, v)$ is $A$-invariant.}{}

\prpc{Given a square matrix $A \in \K^{n\times n}$ and a vector $v \in \K^n$, it holds 
	$$\forall m \quad \dim \mathcal{K}_m(A, v) = m \iff \deg_A(v) \geq m.$$}{}
\corc{$\forall m \quad \dim \mathcal{K}_m(A, v) = \min\{m, \deg_A(v)\}$.}{}

\subsection{Krylov iterations}
Given a linear system $Ax = b$ and an initial guess $x_0 \in \K^n$ any Krylov method search the $m^{th}$ approximate solution in the affine space 
$$x_0 + \mathcal{K}_m(A, r_0) = \{x_0 + p_{m-1}(A)r_0 \mid p_{m-1} \in P^{m-1}\} = \{\(I_n + Ap_{m-1}(A)\)x_0 + p_{m-1}(A)b \mid p_{m-1} \in P^{m-1}\}$$
which implies that the $m^{th}$ residual $r_m$ belongs to the space $r_0 + A\mathcal{K}_m(A, r_0)$.

In the special case of $x_0 = 0$ this approach consist in approximate the solution $x^* = A^{-1}b$ by the the action of a $m-1$ degre polinomial in $A$ on the vector $b$, i.e. in approximate the inverse $A^-1$ with a polinomial in A.
This is reasonable given the Cayley–Hamilton theorem and the following approximation theorems.

\teoc[Weistrass approximation]{Fixed a compact interval $I \subseteq \R$, $\forall f \in C^0(I) \quad \exists \{p_m \in P^m\} \tc \|f - p_m\|_\infty \convIn{m\to \infty} 0$.}{}
\teoc[Bernstein]{Fixed a compact interval $I \subseteq \R$, $\forall f$ analytic on an ellipse $\Omega \subseteq \C$ \st $\Omega \supset I$ $\quad$ $\exists \{p_m \in P^m\}$ and $\alpha >0 \tc \forall m \quad \|f - p_m\|_\infty < Ce^{-\alpha m}$.}{}

\teoc{Given a \spd matrix $A$, the function $f(x) = x^{-1}$ is analytic on an ellipse $\Omega \subseteq \C$ \st $I_A = [\lambda_\min(A), \lambda_\max(A)] \subset \Omega$ and $0 \not \in \Omega$.}{}
\corc{$\exists \{p_m \in \P_m\} \st \|A^{-1} - p_m(A)\|_\infty \convIn{} 0$ exponentially.}{}
\corc{If $A = Q\Lambda Q^H$ then $\forall b \quad \|p_m(A)b - A^{-1}b\|_2 \leq C'e^{-\alpha m}\|b\|_2$.}{}
%\dimc{$\forall b \quad 
%	{\begin{array}{rl}{\|p_m(A)b - A^{-1}b\|_2 &\leq \|p_m(A) - A^{-1}\|_2\|b\|_2 \\
%										&= \|p_m(A) - \Lambda^{-1}\|_2\|b\|_2 \\
%										& = \(\mx_i\left|p_m(\lambda_i(A)) - \frac{1}{\lambda_i(A)}\right|\)\|b\|_2 \\
%										& \leq \(\mx_{\lambda \in I_A}\left|p_m(\lambda) - \frac{1}{\lambda}\right|\)\|b\|_2 \\
%										& \leq C'e^{-\alpha m}\|b\|_2\end{array}}$.}

Since the $m^{th}$ is defined inside a $m$-dimensional subspace, at each step $m$ independent conditions are needed to have an unique solution. This is achieved by the following   

\teoc[Saad]{Given linear system $Ax = b$, an initial guess $x_0$ and an $m$ \st $\dim\mathcal{K}_m(A, r_0) = m$, if \\
	$\bullet$ (C) $\quad$ $A = A^H$ and $\mathcal{C}_m = \mathcal{K}_m(A, r_0)$ \\
	or \\
	$\bullet$ (M) $\quad$ $A$ is non singular and $\mathcal{C}_m = A\mathcal{K}_m(A, r_0)$ \\
	then $\exists! x_m \in x_0 + \mathcal{K}_m(A, r_0) \st x_m \perp \mathcal{C_m}$. \\
	Therefore this unique vector satisfies \\
	$\|x_m - x^*\|_A = \min_{z \in x_0 + \mathcal{K}_m(A, r_0)} \|z - x^*\|_A = \min_{p \in \Pi_m} \|p(A)e_0\|_A$ in the (C) case, and \\
	$\|b - Ax_m\|_2 = \min_{z \in x_0 + \mathcal{K}_m(A, r_0)} \|b - Az\|_2 = \min_{p \in \Pi_m} \|p(A)r_0\|_2$ in the (M) case.}{}



\end{document}